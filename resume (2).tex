\documentclass[a4paper,10pt]{article}
\usepackage{anysize}
\usepackage[space]{grffile}
\usepackage{amsmath}
\usepackage{amssymb}
\usepackage{graphicx}
\usepackage[left=0.75in, right=0.75in, top=0.5in, bottom=0.75in, includefoot, headheight=13.6pt]{geometry}
\usepackage{color,graphicx}
\usepackage{verbatim}
\usepackage{hyperref}
\usepackage{multirow}
\usepackage{latexsym}
\usepackage{mdwlist}
\usepackage{tabularx}
\renewcommand{\labelitemii}{$\circ$}
\renewcommand{\baselinestretch}{1.15}



\hypersetup{
bookmarks=true, 
unicode=false, 
pdftoolbar=true, 
pdfmenubar=true,
pdffitwindow=true,
pdftitle={resume},
 pdfauthor={Newton}, 
pdfsubject={Placements IITTP},
colorlinks=true,
linkcolor=magenta,
citecolor=blue,
filecolor=magenta,
urlcolor=cyan
}


\addtolength{\oddsidemargin}{-0.215in}
\addtolength{\textwidth}{0.2in}
\definecolor{titleColor}{rgb}{0.85, 0.85, 0.85}

\begin{document}

\begin{table}[h!]

\begin{center}
\begin{tabular}{ p{1.05in}p{4.45in}p{0.8in}}
\raisebox{-1.05\totalheight}{\includegraphics[width=1.47in]{{"<LOGO NAME WITHOUT EXTENSION>"}.png}}
&
\begin{itemize}
\setlength\itemsep{.01em}
\item[] \textbf{Addu Mahanthi}
\item[] \textbf{B.Tech in Civil and Environmental Engineering, Indian Institute of Technology Tirupati, India}
\item[] \textbf{Indian Institute of Technology Tirupati, India}
\item[] \textbf{\url{MahanthiAddu(https://www.linkedin.com/in/ mahanthi-addu-b3a17b1730)}}
\end{itemize}
&
\raisebox{-0.8\totalheight}{\includegraphics[width=1in,height=1.3in]{{"<YOUR PHOTO NAME WITHOUT EXTENSION>"}.png}}
\end{tabular}
\end{center}
\end{table}

\vspace{-.8cm}

\colorbox{titleColor}{\parbox{6.7in}{\textbf{Education Details}}}
\\ \\
\indent \begin{tabular}{p{2.2in} p{2.6in} p{0.5in} p{0.9in}}
\hline
\textbf{Program} & \textbf{Institute} & \textbf{Year} & \textbf{\%/CGPA} \\ 
 \hline

Bachelor of Technology & IIT Tirupati & 2021 & 7.61\\ 
Intermediate & Narayana Junior College & 2017 & 98.6\%\\ 
Secondary School Education & Sri Aurobindo Kakatiya Secondary School  & 2015 & 9.8\\ 
 &  &  & \\ 
\end{tabular}\\

\colorbox{titleColor}{\parbox{6.7in}{\textbf{Technical Proficiency}}}\\ 

\begin{tabular}{p{1.6in}p{0.1in}p{4.5in}}
\textbf{\small{Software/Tools}} &: &{{AutoCAD, Revit, Staad.Pro, QGIS, openLCA, SimaPro}} \\
\textbf{\small{Programming Languages}} &: &{{Python, C}} \\
\end{tabular}\\

\colorbox{titleColor}{\parbox{6.7in}{\textbf{Experience}}}

\begin{itemize*}
\setlength{\itemsep}{1pt}
\item \textbf{Subject Matter Expert at Ace Engineering Academy}\hfill {\small{{\textbf{[6 months]}}\/}}
\begin{itemize*}

            \item Working as Content Developer in Civil Engineering department. 

            \end{itemize*}

            \end{itemize*}

\colorbox{titleColor}{\parbox{6.7in}{\textbf{Projects}}}

\begin{itemize*}
\setlength{\itemsep}{1pt}
\item \textbf{San-Fransico Bridge}
 \\ {(\textbf{Guide :} Professor Dr. Krishnaiah)}\hfill {\small{{\textbf{[July - Dec '17]}}\/}}
\begin{itemize*}
\setlength{\itemsep}{.00pt}

            \item \textbf{Abstract}: Abstract: The Bridge was made with straws, sticks and threads. The model of bridge was made to
understand strength and capacity , when the bridge is made of light weight material. 

            \end{itemize*} 

            \end{itemize*} 

\begin{itemize*}
\setlength{\itemsep}{1pt}
\item \textbf{Building Design and Drawing}
 \\ {(\textbf{Guide :} Dr. Kalai Selvi)}\hfill {\small{{\textbf{[July - Dec ' 18]}}\/}}
\begin{itemize*}
\setlength{\itemsep}{.00pt}

            \item \textbf{Abstract}: Abstract: Residential Building was designed for chosen location following principles of functional design
of buildings using Autocad and Revit.
 

            \end{itemize*} 

            \end{itemize*} 

\begin{itemize*}
\setlength{\itemsep}{1pt}
\item \textbf{Effects of Lane width, Shoulder width on Highway Safety}
 \\ {(\textbf{Guide :} Dr. Gowri)}\hfill {\small{{\textbf{[July - Dec ' 19]}}\/}}
\begin{itemize*}
\setlength{\itemsep}{.00pt}

            \item \textbf{Abstract}: Abstract: A Linear model was generated with the lane width, shoulder width and number of accidents
data to find the best suitable combination of lane and shoulder width to avoid accidents on Highways. 

            \end{itemize*} 

            \end{itemize*} 

\begin{itemize*}
\setlength{\itemsep}{1pt}
\item \textbf{Specific Yield of a Location}
 \\ {(\textbf{Guide :} Dr. Prasanna V Sampath)}\hfill {\small{{\textbf{[July - Dec ' 19]}}\/}}
\begin{itemize*}
\setlength{\itemsep}{.00pt}

            \item \textbf{Abstract}: Abstract: Using the Rainfall data, Groundwater level data provided by APSDPS website , Specific yield
of a selected location is calculated and verified whether it is reasonable for the kind of soil found in the
region.
 

            \end{itemize*} 

            \end{itemize*} 

\begin{itemize*}
\setlength{\itemsep}{1pt}
\item \textbf{Environmental Impacts of a Construction project using Life Cycle Approach - A case study in IIT Tirupati.}
 \\ {(\textbf{Guide :} Dr. Suresh Jain)}\hfill {\small{{\textbf{[Sep ’ 20 - May ’ 21]}}\/}}
\begin{itemize*}
\setlength{\itemsep}{.00pt}

            \item \textbf{Abstract}: Abstract: A real life building is taken into account for assessing the potential environmental impacts
using Life Cycle Assessment. The study is limited to the construction phase of a construction project
including transportation of materials to the site. ReCipe approach is used to depict environmental effects
at midpoint and endpoint levels. 

            \end{itemize*} 

            \end{itemize*} 

\colorbox{titleColor}{\parbox{6.7in}{\textbf{Relevant Courses}}}\\[0.08in]
    \begin{tabular}{p{3.5in}p{3in}p{2.5in}}
\hspace{0.9pc}$\bullet$ Transportation Engineering&$\bullet$ Environmental Engineering\\[0.05in]
\hspace{0.9pc}$\bullet$ Estimation and Construction Management&$\bullet$ Project Planning and Control\\[0.05in]
\hspace{0.9pc}$\bullet$ Functional Design of Buildings&$\bullet$ Traffic Engineering and Road Safety\\[0.05in]
\hspace{0.9pc}$\bullet$ Organizational Behavior&$\bullet$ Leadership and Team Management\\[0.05in]
\hspace{0.9pc}$\bullet$ Professional Ethics&$\bullet$ Principles of Economics\\[0.05in]
\end{tabular}


\colorbox{titleColor}{\parbox{6.7in}{\textbf{Positions of Responsibility}}}\\

\textbf{Core member of CREATIVE CELL}  \hfill {\small{{\textbf{[Dec ’ 18 - Feb ’ 19]}}}\/} 
\begin{itemize*} 

        \item Responsibility of enriching college premises for the annual fest with hand made creative objects. 

        \end{itemize*}

\textbf{Co - coordinator of ARTISTA CLUB}  \hfill {\small{{\textbf{[Aug ’ 19 - May ’ 20]}}}\/} 
\begin{itemize*} 

        \item Responsibility of organizing art events and competitions 

        \end{itemize*}

\colorbox{titleColor}{\parbox{6.7in}{\textbf{Extra Curricular activities}}}

\begin{itemize}

    \setlength{\itemsep}{1pt}
\item Participated in art events like Pencil sketching, Glass painting, Costume Designing. \hfill 
\item Participated in dance events like Flash mob, Stage perfomances. \hfill 
\item Participated in Inter IIT sports meet, representing IIT Tirupati in women basket ball \hfill 
\end{itemize}

\colorbox{titleColor}{\parbox{6.7in}{\textbf{Hobbies and Interests}}}

\begin{itemize}
        \setlength{\itemsep}{1pt}
\item Playing Chess, Basketball
\end{itemize}

\begin{itemize}
        \setlength{\itemsep}{1pt}
\item Origami and Quilling art.
\end{itemize}

\begin{itemize}
        \setlength{\itemsep}{1pt}
\item Pencil Sketching.
\end{itemize}

\begin{itemize}
        \setlength{\itemsep}{1pt}
\item Dancing
\end{itemize}

\end{document}
